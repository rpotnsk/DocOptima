\section{Описание раздела <<Визиты>>}
\begin{enumerate}[\thesection .1]
\item Раздел обеспечивает доступ к следующим функциям: 
\begin{itemize}
	\item Просмотр маршрута на определённую дату;
	\item Ввод результата визита (изменения статуса визита);
	\item Создание документа;
	\item Удаление внеплановой точки маршрута;
	\item Просмотр детальной информации по точке маршрута;
	\item Просмотр списка документов, созданных в определённой точке маршрута;
	\item Просмотр детальной информации документа, созданного в определённой точке маршрута;
	\item и т.д;
\end{itemize}
\item В основном окне модуля отображается список торговых точек на заданную дату (по умолчанию текущую) для заданного торгпреда (владельца планшета). Окно содержит информационную строку, область фильтров и область списка торговых точек маршрута.
В информационной строке указывается количество посещённых точек относительно общего количества из маршрута на установленную дату. 
В области фильтров могут быть установлены следующие фильтры: 
	\begin{itemize} 
	\item Дата – дата, за которую отображается маршрут.
	\item Статус – статус маршрута (Не посещён, Посещён, Все).
	\end{itemize}		  
\item В списке визитов отображаются элементы, каждый из которых предоставляет краткую информацию по визиту, а именно: 
\begin{itemize}
	\item Название точки;
	\item Адрес точки;
	\item Результат посещения;
	\item Признак внепланового визита.
\end{itemize}		
 	
\item При удержании элемента маршрута (визита) появляется контекстное меню, в котором можно выполнить следующие операции: 
\begin{itemize}
	\item Изменить статус визита (например, проставить причину отказа);
	\item Создать документ;
	\item Создать оптимальный документ;
	\item Удалить маршрут (удалить внеплановую точку маршрута);
	\item Запустить сценарий.
\end{itemize}	
\item Запись о визите выделяется красным цветом в общем случае, если у клиента имеется просроченная задолженность. 
\end{enumerate}