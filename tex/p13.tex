\section{Описание раздела <<Документы>>}
\begin{enumerate}[\thesection .1]
\item Раздел предназначен для работы с документами.
В модуле выполняются следующие операции:
\begin{itemize}
	\item Просмотр списка документов;
	\item Просмотр детальной информации по документу;
	\item Редактирование документа;
	\item Создание документа по образцу; 
	\item Удаление документа;
	\item Простановка / удаление отметки акцептации.
\end{itemize}
\itemВ главном окне модуля отображается: информационная строка (в верхней части экрана), область фильтров и список сформированных документов (по умолчанию за текущую дату). 
В модуле доступны следующие фильтры: 
	\begin{itemize} 
	\item Дата – фильтр по дате. Фильтр может быть неизменяемым при вызове из модуля «Визиты».
	\item Контрагент – фильтр по ТТ документа. Фильтр может быть неизменяемым при вызове из модуля «Визиты».
	\item Тип – фильтр по типу документа (заказ, дистрибьюция и т.д.).
	\end{itemize}		  
\item В списке документов отображаются следующие параметры документа:
\begin{itemize}
	\item Номер;
	\item Дата и время;
	\item Для документов типа «Заказ» - дата поставки (выделена красным цветом);
	\item Тип документа;	
	\item статус передачи документа – символ, означающий следующее:
	\begin{itemize}
		\item «+» – документ акцептован, но не передан на сервер;
		\item «*» – документ передан на сервер;
		\item «\#» – документ передан на сервер и обработан;
		\item « » (отсутствие символа) – документ не акцептован.
	\end{itemize}				
	\item Клиент;
	\item Имя контрагента, который сформировал документ;
	\item Адрес клиента - отображение адреса клиента;
	\item сумма документа (подсчитывается в тех случаях, когда имеется смысл в её подсчете, например, для документов типа «Заказ»);	
\end{itemize}		
\item Для того чтобы отредактировать документ необходимо в модуле «Документы» вызвать контекстное меню (путем удержания редактируемого документа) и выбрать пункт «Изменить документ». В случае если документ акцептован, то Система выдаст информационное окно с текстом «Редактирование документа невозможно» и кнопкой «ОК». Если у документа снята отметка акцептации – то будет открыто окно редактирования документа. 
В режиме редактирования документа пользователю доступны те же операции, что и при создании документа (см. раздел \ref{sec:sec3_1} ). 


\end{enumerate}